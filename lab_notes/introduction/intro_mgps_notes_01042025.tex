\section{Introduction}
\subsection{What is Trace Evidence Analysis?}
\begin{itemize}
    \item A forensic discipline that examines tiny material transfers (e.g., hair, fibers, soil) between objects, people, and environments.
    \item These transfers can link individuals to specific locations, objects, or events.
\end{itemize}

\subsection{Challenges in Tracing Geographic Origins of Organisms}
\begin{itemize}
    \item Identifying biological material to link individuals with locations is difficult due to:
    \begin{itemize}
        \item Complexity of biological interactions.
        \item Lack of dynamic data on recent movements.
    \end{itemize}
    \item Human DNA is static—it identifies origins but doesn't track recent movements.
    \item Eran’s GenoChip tool.
\end{itemize}

\subsection{Importance of Microbiomes}
\begin{itemize}
    \item Microbiomes (bacteria, fungi, viruses, etc.) provide dynamic, spatiotemporal information that changes with an individual’s environment.
    \item Microbiomes can be used as biogeographical markers for tracing movements.
\end{itemize}

\subsection{Broader Applications of Microbial Biogeography}
\begin{itemize}
    \item \textbf{Forensics}: Predicting the geographic origin of samples.
    \item \textbf{Ecology}: Understanding biodiversity and environmental influences on microbial communities.
    \item \textbf{Medicine}: Addressing the spread of antimicrobial resistance (AMR), a major global challenge.
    \item \textbf{Policy Development}: Informing guidelines to reduce AMR risks in human mobility, trade, and food distribution.
\end{itemize}

\subsection{Antimicrobial Resistance (AMR) and Microbial Tracing}
\begin{itemize}
    \item AMR is spread through human travel, migration, and global trade of goods (e.g., food and animals).
    \item Example: AMR bacteria spread through trade routes, like illegally traded species, can be traced using microbiome data.
\end{itemize}

\subsection{Introducing mGPS}
\begin{itemize}
    \item A machine learning–based tool to identify the fine-scale geographic origin of microorganisms using microbial relative sequence abundances (RSAs).
    \item Tested on urban, soil, and marine microbiomes.
    \item mGPS successfully distinguishes local from nonlocal microorganisms and traces global AMR gene spread.
\end{itemize}
\subsection{Why this matters?}
\begin{itemize}
    \item Understanding microbial biogeography helps manage AMR risks, improve forensics and epidemiology and study human environment interactions.
\end{itemize}